\begin{appendix}

\lstset{ % General settings
language=c++,                   % choose the language of the code
basicstyle=\ttfamily \small,    % the size of the fonts that are used for the code \footnotsize
numbers=left,                   % where to put the line-numbers
numberstyle=\small,             % the size of the fonts that are used for the line-numbers
stepnumber=2,                   % the step between two line-numbers. If it's 1 each line will be numbered
numbersep=10pt,                 % how far the line-numbers are from the code
showspaces=false,               % show spaces adding particular underscores
showstringspaces=false,         % underline spaces within strings
showtabs=false,                 % show tabs within strings adding particular underscores
frame=,                         % adds a frame around the code (single)
tabsize=2,                      % sets default tabsize to 2 spaces
captionpos=t,                   % sets the caption-position: top (t), bottom (b)
breaklines=true,                % sets automatic line breaking
breakatwhitespace=false,        % sets if automatic breaks should only happen at whitespace
escapeinside={\%*}{*)},         % if you want to add a comment within your code
caption=footnote, 
label=listing:relRef
}

\section{Code}
A code and data will be distributed using git and made abvailable 
at \\ \url{http://www.helsinki.fi/~miheikki/system/refs/heikkinen/ah09bProceedings/code}

\subsection{Makefile}
\lstset{
language=bash,
numbers=left,
stepnumber=2,
caption={\tt code/Makefile},
label=makefile
}
\lstinputlisting{code/Makefile}


\subsection{tmva-common.conf}
\lstset{
language=bash,
numbers=left,
stepnumber=2,
caption={\tt code/tmva-common.conf},
label=tmvacommonconf
}
\lstinputlisting{code/tmva-common.conf}

\subsection{tmva-example.conf}
\lstset{
language=bash,
numbers=left,
stepnumber=2,
caption={\tt code/tmva-example.conf},
label=tmvacommonconf
}
\lstinputlisting{code/tmva-example.conf}

\subsection{ametisti.sh.job}
\lstset{
language=bash,
numbers=left,
stepnumber=2,
caption={\tt code/ametisti.sh.job},
label=ametistishjob
}
\lstinputlisting{code/ametisti.sh.job}

\subsection{chep09tmva.C}
\lstset{
language=C++,
numbers=left,
stepnumber=2,
caption={\tt code/chep09tmva.C},
label=chep09tmvac
}
\lstinputlisting{code/chep09tmva.C}

\subsection{chep09tmva.cc}
\lstset{
language=C++,
numbers=left,
stepnumber=2,
caption={\tt code/chep09tmva.cc},
label=chep09tmvacc
}
\lstinputlisting{code/chep09tmva.cc}

\section{Data files}
Minimalistic example datafile (recommendation: less than 1-2 MB) can
be included in the repository for testing and demonstration
purpooses. Although Git can easily deal with large files, it is
recommended that the production data would not be
included in the repository. It should be kept in mind that GitHub
(and shell accounts) offer relatively limited disk space (GitHub: 100
MB, CERN default: about 150 MB) and that ROOT files can probably not be
compressed further.

For production use it is recommended to store in the repository an URL
to the data. Then we can use the {\tt ROOT} or {\tt HTTP} protocols to
access it or use a {\tt make} directive/shell script to copy the data to the
user's computer. One good example of this practice is the {\tt
HipProofAnalysis} repository. Only the URL:s are stored, not the
data. Additionally also the parameters, configuration options,
software versions, etc. used for the production of the data should
probably be stored in some way in the repository.

\section{WORKING NOTES}
{\bf Suggested  responsibility}: 
\begin{itemize}
\item[aatos]
Aatos: editor, NN classifiers; 
\item Pekka: git consulting, PROOF
\begin{itemize}
\item git consulting: OK (setting up workflow and repositories, user
  training, documentation, software installation)
\item PROOF: I didn't see PROOF mentioned anywhere in the TMVA
  documentation. Is it supported? If not, then I don't have resources
  to do it (lesson learned in the past: PROOF-enabling an analysis
  code can be a major undertaking...)
\end{itemize}
\item Sami: MC data, 
\item Lauri 1-prog physics
\item Ritva: 
\item Tomas: Ametisti
\item Tapio: 
\item Matti:a mechanism to work with variables 
\item Veikko:
\end{itemize}
\subsection{HISTORY}
\begin{itemize}
\item 081216 Merge from Matti and Pekka.
\item 081202 Merging example data and related configuration file from Lauri.
\item 081125 Merging from Lauri, Matti, and Pekka. 
Added subsections for code listing and table of contents.
\item 081111 Merging branch from Lauri and including comments from Sami. 
Based on discussion at HIP group weekly meeting made some aditional changes to abtract.
\item 081028 Project released in \url{http://github.com/aatos/chep09tmva}. Removed proceedings notes in the Appendix A to separate file {\tt notes.tex}.
\item 091029 PK: Commented some points in the proposed
  responsibilities. Added a couple of links to the Git documentation.
\item 081021 Title and abstract focus improved after discussion in the group. 
\item 081014 First draft done after the idea to have TMVA paper at next CHEP was accepted in the group.
\end{itemize}

To be done:
\begin{itemize}
\item Template code for  analysis using latest ROOT, and TMVA inside it.
\item Revise title, abtract and paper structure including appendix.
\end{itemize}

\subsection{Code repository}
\begin{itemize}
\item Source code for paper and TMVA script  is available at 
{\tt git://github.com/aatos/chep09tmva.git} (\url{http://github.com/aatos/chep09tmva})
%git commit -a -m 'comment'  (add all changes)
%git push 

\begin{itemize}
\item Lauri:
\begin{verbatim}
git remote add lauri http://cmsdoc.cern.ch/~wendland/chep09tmva.git

git fetch lauri
git merge lauri/master
\end{verbatim}
\item Pekka:
\begin{verbatim}
git remote add pekka git://github.com/kaitanie/chep09tmva.git

git fetch pekka
git merge pekka/master
\end{verbatim}
\item Matti:
\begin{verbatim}
git remote add matti git://github.com/makortel/chep09tmva.git

git fetch matti
git merge matti/master
\end{verbatim}
\end{itemize}

\item Alternatively LaTeX-files can be loaded form
\url{http://www.helsinki.fi/~miheikki/system/refs/heikkinen/ah09bProceedings.tar.gz}.
\item You can also mail you comments and updates directly to editor (Aatos)
Based on pdf version
\url{http://www.helsinki.fi/~miheikki/system/refs/heikkinen/ah09bProceedings.pdf}.
\end{itemize}

Guide \url{http://ktown.kde.org/~zrusin/git/git-cheat-sheet-medium.png}.

Some git documentation:
\begin{itemize}
\item Git tutorial: \url{http://www.kernel.org/pub/software/scm/git/docs/gittutorial.html}
\item Git with HipProofAnalysis (contains instructions on how to use
  Git on lxplus:
  \url{http://projects.hepforge.org/radical/trac/wiki/GitWithHipProofAnalysis}
\end{itemize}

\subsection{Building the document}

Building the document requires {\tt make} and \LaTeX tools. The
document can be built using the {\tt make} command. At the end of the
compilation this will optionally launch a PDF viewer (by default Firefox browser
and Acrobat Reader plugin). You can change the PDF viewer program by
setting environment variable {\tt PDFVIEWER} to point to your
favourite PDF viewer (e.g. lightweight alternative {\tt xpdf}). To
enable the PDF viewer feature you can set the environment
variable {\tt USEVIEWER} to 1.

\subsection{Current status of TMVA}
For introduction browse, six talks from year 2008 \url{http://tmva.sourceforge.net/talks.shtml}.

\begin{itemize}
\item Current version is TMVA-v3.9.5 (2008 Aug. 9th).
\item TMVA (\url{http://tmva.cvs.sourceforge.net}) is now released as ROOT package

\begin{itemize}
\item ROOT version from 5-19-02a to 5-21-01-alice contains TMVA 3.9.4.
\item ROOT version 5.22 is planned to be released on December 18, 2008
  (release notes \url{http://root.cern.ch/root/v522/Version522.news.html}),
  it has TMVA-v.3.9.5
\end{itemize}

\item In addition to many bug fixes:
\begin{itemize}
\item Improved prepossessing
\item Pre-selection cuts on arrays. Previously used {\em TEventlists} 
(only event  wise pass/fail) were replaced by {\em TreeFormulas} (sensitive to array position).
\item Plugin capability: custom multivariate classifier can now be plugged into
    the TMVA framework to benefit from TMVA's analysis and performance comparison
    tools. 

\item For details see release notes 
\url{http://tmva.cvs.sourceforge.net/*checkout*/tmva/TMVA/development/RELNOTES}
\end{itemize}

\end{itemize}



\subsection{TMVA run configuration files}

The new example program ({\tt code/chep09tmva.cc}) uses a config file
({\tt code/tmva.conf}) for classifier configuration. There is one
possible problem in this setup. If everyone edits the same file time
and time again, merging everyone's work will become very painful. This
is a problem because we would like people to merge early and
often. There are a few proposals that should be investigated as
possible solutions to this problem:
\begin{enumerate}
\item Using config files is a good option. Hardcoding configs into
the program would probably make merges quite difficult as well.
\item Each user/classifier has a separate config file. The {\tt
chep09tmva} program should have a command line option that allows the
user to choose which configuration is used. An example invocation of
the {\tt chep09tmva} program is shown in listing \ref{configExample}.
\item Ability to have common config options in a separate file
(e.g. {\tt tmva-common.conf}) which could be included into
user/classifier specific configuration files with an {\tt include}
statement. An example of this is shown in listings \ref{commonConfig}
and \ref{userConfig}.
\end{enumerate}

The program has been modified as follows
\begin{itemize}
\item Support for \texttt{include} as shown in listing
  \ref{commonConfig}
\item There is now a common configuration file
  \texttt{tmva-common.conf} (which is still more to demonstrate than
  to really do anything useful), and an example of user configuration
  \texttt{tmva-example.conf}
\item By default it uses the \texttt{tmva-common.conf}, but the
  configuration can be specified as shown in listing \ref{configExample}
  \begin{itemize}
  \item If the same directive (\texttt{Variables:}, \texttt{Cuts:},
    \texttt{Trainer:}, \texttt{Classifiers:}) is given in both the
    user configuration and common configuration, the user
    configuration is used (i.e. e.g. variable lists are not merged).
  \end{itemize}
\end{itemize}



\lstset{
language=bash,
numbers=left,
stepnumber=2,
caption=Example invocation of {\tt chep09tmva} with config file name as a parameter.,
label=configExample
}
\begin{lstlisting}
./chep09tmva pekka.conf
\end{lstlisting}

\lstset{
language=bash,
numbers=left,
stepnumber=2,
caption=Contents of the file {\tt tmva-common.conf} that contains config options shared by all analysis runs.,
label=commonConfig
}
\begin{lstlisting}
// String to pass TMVA::Factory::PrepareTrainingAndTestTree
Trainer:
NSigTrain=1000:NBkgTrain=20000:SplitMode=Random:NormMode=NumEvents:!V
\end{lstlisting}

\lstset{
language=bash,
numbers=left,
stepnumber=2,
caption=Contents of the user specific config file {\tt pekka.conf}.,
label=userConfig
}
\begin{lstlisting}
include tmva-common.conf

Cuts_D H:!V:FitMethod=MC:EffSel:SampleSize=20000:VarProp=FSmart:VarTransform=Decorrelate
\end{lstlisting}

\lstset{ %
language=bash,                % choose the language of the code
basicstyle=\footnotesize,       % the size of the fonts that are used for the code
numbers=left,                   % where to put the line-numbers
numberstyle=\footnotesize,      % the size of the fonts that are used for the line-numbers
stepnumber=2,                   % the step between two line-numbers. 
                                %If it's 1 each line will be numbered
numbersep=5pt,                  % how far the line-numbers are from the code
showspaces=false,               % show spaces adding particular underscores
showstringspaces=false,         % underline spaces within strings
showtabs=false,                 % show tabs within strings adding particular underscores
frame=single,                   % adds a frame around the code
tabsize=2,                      % sets default tabsize to 2 spaces
captionpos=t,                   % sets the caption-position to bottom
breaklines=true,                % sets automatic line breaking
breakatwhitespace=false,        % sets if automatic breaks should only happen at whitespace
escapeinside={\%*}{*)},          % if you want to add a comment within your code
%caption=Bash function to release a directory., 
label=listing:relRef
}

\end{appendix}
