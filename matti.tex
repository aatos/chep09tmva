\subsection{[Matti] SVM}

\subsubsection{General notes}
\begin{itemize}
\item Number of bins for the ROC curve is 100, and specified in
  \texttt{src/Config.cxx} under the TMVA source tree.
  \begin{itemize}
  \item This can be changed by \texttt{TMVA::gConfig().GetVariablePlotting().fNbinsXOfROCCurve = 1000;}
  \item It looks like when the ROC curve is constructed (in
    \texttt{src/MethodBase.cxx}; except for Cuts), the background
    efficiency is interpolated with splines from a histogram with
    $10^4$ bins. The histogram in question is the bacgkround
    efficiency as a function of discriminator value
  \item It could be possible to recompute the ROC curve in a plotting
    script (except for Cuts)
  \item The whole thing is probably not a problem, since the
    background efficiency is floating point number
  \end{itemize}
%\item All variables (branches) in the tree are float, right? \textbf{Yes}
\item Signal and background event weights are now both 1. This should
  probably be fixed; is hardcoding to \texttt{code/chep09tmva} ok or
  do we want it to be a configuration parameter?
\item The normalization mode should be checked(NormMode parameter for
  the trainer, currently it has the value NumEvents), also the
  SplitMode (now Random) might not be good
%\item In the case that the big data is in madhatter, we should use
%  ametisti (or sepeli). I think we should at least discuss about the
%  possibility of having the data in ametisti instead of madhatter.
\item The \texttt{TestTree} tree in output \texttt{TMVA.root} can be
  used to see the output distributions of the variables after placing
  a cut on the classifier discriminator
  \begin{itemize}
  \item The branch \texttt{type} seems to be 0 for background and 1 for signal
  \end{itemize}
\item What number(s) should we look when optimizing the classifier?
  \begin{itemize}
  \item Signal efficiency at $10^5$ background rejection?
  \end{itemize}
\item Jets vs. events
  \begin{itemize}
  \item The input \texttt{TTree} for TMVA has jets as entries (i.e.
    TMVA ``event'' corresponds to a jet in the event generated in
    Pythia), so the efficiencies (and other quantities) reported by
    TMVA are for jets. The tree has branches for event and run
    numbers, so it is possible to separate the jets belonging to
    different events.
  \item In the end, in order to have a physically reasonable result
    and to compare with other results, we would like to have the
    signal/background efficiencies of events, not jets (e.g. event is
    passed if there is at least one jet passing the classifier,
    otherwise the event is rejected)
  \item In order to ``convert'' the jet efficiencies to event
    efficiencies, some kind of counting must be done
    \begin{enumerate}
    \item[a)] Optimisation is done only with jets, and for the final
      results the classifiers are used in event basis. Our abstract
      doesn't clearly specify, what we mean by $10^5$ background
      rejection (event vs. jet rejection). 

    \item[b)] Use TMVA training tree for re-counting. However, the
      training tree contains only the variables used for training, and
      classifier outputs for the corresponding jets. In the current
      TMVA (v3.9.x) it seems to be impossible to add variables, which
      are not used for training/testing, to the training tree.

      It is of course possible to take the standalone TMVA as a part
      of our package and include this functionality. Actually, this
      has been requested in the TMVA users mailing list in 2007
      (\url{http://sourceforge.net/mailarchive/forum.php?thread_name=471C8D74.6040209\%40mpi-hd.mpg.de\&forum_name=tmva-users}),
      and it is in the TODO list
      (\url{http://tmva.cvs.sourceforge.net/tmva/TMVA/development/todo.txt?view=markup})
      with \emph{high priority}. Thus, my motivation of implementing
      this feature is not so high.

    \item[c)] Use the TMVA input tree for performing our own
      event-based counting. The TMVA preselection cuts and variable
      transformations will cause some headache, but it could be
      manageable. It might be easiest to take relevant pieces of code
      from TMVA and add glue, tape and gum to make it work. In order
      to not to use training data for testing, the training and
      testing trees should be in different files.
    \end{enumerate}
  \end{itemize}
\end{itemize}

\subsubsection{SVM Notes}
\begin{itemize}
\item Training time scales as $O(n^2)$ where $n$ is the size of the
  input data (i.e. number of events)
  \begin{itemize}
  \item 100 signal and 200 background events took 0.125 seconds
  \item 200 signal and 400 background events took 0.542 seconds
  \item 500 signal and 1000 background events took 3.41 seconds
  \item 800 signal and 1600 background events took 10 seconds
  \item 1000 signal and 2000 background events took 13.8 seconds
  \item If the scaling law is correct, $10^4$ events would take about 2
    minutes, $5\times 10^4$ events would take almost an hour and $10^5$
    events would take more than 3,5 hours.
  \end{itemize}
\end{itemize}

